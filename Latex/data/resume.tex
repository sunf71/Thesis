\begin{resume}

  \resumeitem{个人简历}

  1983 年 7 月 1 日出生于 江西 省 景德镇市。

  1999 年 9 月考入 解放军理工大学 通信工程学院  通信工程 专业,2003 年 6 月本科毕业并获得 工学 学士学位。

  2003 年 9 月考入 解放军理工大学 通信工程学院 通信与信息系统专业, 2006年2月研究生毕业并获得工学硕士学位。

  2006 年 3月至2011年8月 在中国卫星海上测控部工作,历任助理工程师,工程师。

  2011年9月考入清华大学计算机系攻读博士学位至今。



  \researchitem{发表的学术论文} % 发表的和录用的合在一起

  % 1. 已经刊载的学术论文(本人是第一作者,或者导师为第一作者本人是第二作者)
  \begin{publications}
    \item {\bf Feng Sun}, Kaihuai Qin, Wei Sun. Fast Two-Step Exemplar-Based Image Inpainting,
   The Seventh International Conference on Image and Graphics (ICIG), 2013. (EI收录, 检索号: 20140117162015)
    \item  {\bf Feng Sun}, Kaihuai Qin, Wei Sun, Huayuan Guo. Fast Background Subtraction Based on Non-parametric Models for Freely Moving Cameras,
    Computer Graphics International (CGI), 2015.
    \item {\bf 孙丰}, 秦开怀, 孙伟, 郭华源. 一种针对移动相机的实时视频背景减除算法, 计算机辅助设计与图形学学报, 2016, 28(4): 572-578 (EI收录,检索号: 20161902360463)
  \end{publications}

  % 2. 尚未刊载,但已经接到正式录用函的学术论文(本人为第一作者,或者
  %    导师为第一作者本人是第二作者)。
  \begin{publications}[before=\publicationskip,after=\publicationskip]

    \item {\bf 孙丰}, 秦开怀, 孙伟, 郭华源. 基于区域合并的图像显著性检测, 计算机辅助设计与图形学学报, (已录用, EI源刊)
    \item  {\bf Feng Sun}, Kaihuai Qin, Wei Sun, Huayuan Guo. Fast Background Subtraction for Moving Cameras Based on Non-parametric Models,
    Journal of Electronic Imaging, (已录用, SCI源刊, 影响因子:0.672).
  \end{publications}

  % 3. 其他学术论文。可列出除上述两种情况以外的其他学术论文,但必须是
  %    已经刊载或者收到正式录用函的论文。
  \begin{publications}
    \item  Kaihuai Qin, Chun Yang, {\bf Feng Sun}. Generalized Frequency-domain SAFT for Ultrasonic Imaging of Irregularly Layered Objects,
    IEEE Transactions on Ultrasonics, Ferroelectrics, and Frequency Control, 2014, 61(1): 166-179. (SCI源刊, 检索号: 000329522400011, 影响因子: 1.512)




  \end{publications}



  \researchitem{研究成果} % 有就写,没有就删除
  \begin{achievements}
    \item 秦开怀, 杨春, {\bf 孙丰}. 基于变波速相位迁移的多层物体无损检测超声成像方法. 国家发明专利申请号:201210046151.1, 2012.2.24. (已授权)
  \end{achievements}

\end{resume}
