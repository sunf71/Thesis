\chapter{基于区域合并的图像显著性分析算法}
\label{cha:SGRM}

\section{研究背景}
\label{sec:background}
图像显著性区域检测是近年来计算机视觉和图像处理领域的研究热点之一~\cite{saliencySurvey}。该技术的目标是快速定位图像中显著性对象,已经用于目标识别,基于内容的图像编辑以及图像检索等应用。
早期的显著性检测算法[1]依据视觉感知理论,采用自底向上方法预测人眼观察图像时的焦点(Fixation),得到一系列离散的圆点形状的显著性区域预测结果.然而,这类方法效率和准确度均较低,无法得到显著性对象的准确边界。随后,为了实现提前显著性对象的准确边界,研究人员提出了一系列数据驱动,自底向上的显著性区域检测算法[2][3][4][5].这类算法可以较为准确的提取自然图像中包含的显著性对象.据文献[6]报道,在这些算法中,基于图像区域的算法性能更出色.这类算法首先用超像素等图像分割方法将图像分为若干个小区域,随后以区域为单位进行显著性计算. 相比于基于像素的算法, 区域分割处理可以保持对象边界,同时利用区域颜色直方图可以更准确的对区域内的颜色对比度进行量化.另外,以区域为单位计算大大降低了运算量.这些算法的出发点都是根据显著性区域的特点,例如对比度高[3],处在相机焦点上[5]等,定义区域显著性度量公式,最终得到显著性区域结果.然而,自然图像中可能包含各种类型的显著性对象,单纯用这些线索仍然无法处理所有情况.如图1所示,(a)为输入图像,(b),(c),(d)分别为文献[4](RC),文献[6](CSH)算法和本文算法的结果,(e)为正确结果.图1表明,当输入图像中对象和背景的色彩分布较为接近(图1中第一行和第二行),或者背景复杂度较高(第三行)时RC和CSH算法均无法有效检测出显著性对象. 
\section{相关工作
\label{sec:relatedWorks}
\section{算法描述}
\label{sec:algorithm}

\section{实验结果与分析}
\label{sec:results}

\section{本章小结}
