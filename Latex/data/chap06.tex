\chapter{总结与展望}
\label{ch6:Summary}

\section{研究总结}
\label{ch6:sec:summary}
本论文主要研究内容为图像和视频中的背景减除技术,以消除图像或视频中的背景为目标,根据不同应用场合中背景的特点,主要研究了针对自然图像的图像显著性检测问题、包含缺失部分图像的图像填充问题,以及针对移动相机拍摄视频的背景减除问题。本论文的主要研究成果总结如下:
\begin{enumerate}
  \item 为了提高自然图像中显著性区域检测的准确性,特别是针对先有算法在处理包含复杂背景、前景与背景颜色分布较接近等情况的图像时前景检测准确率低的问题,提出了一种基于区域合并的图像显著性区域检测算法。该算法利用自然图像背景区域的特点,以区域合并的方式,逐渐将图像由初始状态下的多个区域合并为包含显著性对象和背景的两个区域。为了防止显著性对象在区域合并的过程中被合并到背景区域,在合并过程的不同阶段采用了不同的合并策略。首先,第一阶段合并中只允许相邻的区域进行合并,利用对象的空间分布的连续性将属于统一对象的区域合并到一起;其次,在第二阶段合并中,处理第一阶段合并过程中产生的遮挡和空洞问题,允许在颜色分布相近但不相邻的区域之间进行合并,同时将小区域合并到其周围最相近的邻居区域中;最后,根据区域显著性分析结果为引导,不断将显著性差的区域合并到背景区域,直到图像被分为前景和背景两个区域。根据在合并过程中得到的多个候选显著性区域,以加权平均的方式得到最终的显著性结果。通过大量实验验证,该算法的显著性区域检测准确度较高,在主要由自然图像组成的ECSSD数据集中得到的准确度和错误率指标优于其他同类算法;

  \item 针对基于样本的图像填充算法保持图像原有结构信息连续性难,计算量大的问题,提出了一种基于多分辨率的两阶段图像填充算法。该算法首先通过DWT对待处理图像进行小波分解,得到低分辨率下的图像。利用DWT 过滤图像高频细节部分保留低频结构信息的特点,以低分辨率图像为引导,以基于样本的图像填充算法对低分率图像和原始图像进行同步填充。为了处理第一步填充结果中在图像边缘部分存在的结构不连续等问题,在这些区域对原始分辨率图像进行第二遍填充,以由粗到精的方式得到最终填充结果。为了保持图像结构信息的连续性,引入梯度结构张量来计算分块填充优先级,并以加权SSD的方式定义图像的最匹配块。在最匹配块的搜索过程中,提出先用分块结构测试的方式对分块的结构相似性进行判断,直接排除结构上差异较大的候选分块,减少冗余计算;同时提出以动态搜索窗口的方式在图像已知区域中搜索最佳匹配块。经过上述优化过程,该算法在速度上相比同类算法有了加大提高,同时可以处理缺失部分面积大、图像结构信息连续差等问题;
  \item 针对移动相机拍摄视频背景减除算法中利用计算密集的图像帧稠密光流或像素点轨迹算法估算相机运动,造成整个算法速度慢的问题,提出了一种基于无参数模型的快速视频背景减除算法。该算法引入ASAPW技术估算并补偿相机运动,为了得到前景对象的准确边界,该算法主要依靠两方面的线索。其中,基于外观的线索来自于基于采样一致性的背景模型。本论文在静止相机情况下的背景模型的基础上,根据移动相机应用场合的不同需求对其进行了改进,并通过并行化编程在GPU上进行了实现;另一方面,基于运动线索对前景对象进行定位,根据稀疏的超像素光流和相机运动的一致性情况得到稀疏的背景种子点,通过区域合并的方式将这些背景区域扩散到整幅图像;最后根据两方面线索得到的两个粗略分割结果,以及前一帧图像得到的前景结果,建立基于超像素的MRF能量函数。通过图割算法求解能量最小化问题,获得最终的前景检测结果。实验结果证明,该算法得到前景结果准确度达到了当前先进算法的水平,但在计算速度方面,该算法要快得多。
  \item 针对在一些应用场合下,例如实时监控、对象检索等,对视频背景减除技术的速度要求更高,准确度要求相对较低的情况,提出了一种针对移动相机的实时背景减除算法。为了实现实时检测,该算法首先利用基于超像素的区域增长预处理方法对视频帧图像进行预处理,得到可能是前景的区域;其次,通过基于相对光流的特征点筛选方法得到来自于背景的特征点,并以分块的方式利用这些特征点估算各分块内的相机运动参数;最后通过验证疑似前景对象的光流与相机运动的一致性得到最终的前景检测结果。该算法简单高效,不需要建立任何背景模型和预处理过程。实验证明该算法能实时处理分辨率为$640 \times 480$的视频,且前景检测准确率高于同类实时算法。
\end{enumerate}

本论文提出的算法较好的解决了图像和视频背景减除应用中的一些技术难题,具有较好的应用前景。然而,对于一些特殊情况,本论文提出的算法并不能很好的处理。本论文研究中存在的主要问题和不足在于:
\begin{enumerate}
  \item 本论文第二章提出的图像显著性检测算法在处理部分背景区域连续性差的图像时,会在区域合并过程中误将前景对象合并到背景区域,无法得到准确的显著性前景;
  \item 本论文第三章中提出的图像填充算法主要考虑恢复图像结构部分的连续性,没有考虑图像中缺失部分主要是平滑纹理区域的情况;
  \item 本论文第四章提出的移动相机视频背景减除算法中,由于使用了基于外观的背景模型,在待处理视频中的第一帧图像已经包括前景对象时会在背景模型初始化阶段将其标记为背景,从而造成处理结果的前面几帧出现错误的前景。随着算法继续运行,通过反馈的模型更新机制这些错误前景会逐渐消失;

  \item 本论文第五章提出的实时背景减除算法利用了移动相机拍摄视频中背景区域连续性的特点。另外假设视频中大部分区域是背景,在视频中背景部分连续性不佳或者大部分区域是前景等情况下会造成算法失败。

\end{enumerate}
\section{未来展望}
\label{ch6:sec:futureWorks}
针对本论文研究当中存在的问题和不足之处,结合对当前技术发展的趋势的判断,未来研究工作的一些设想和展望如下:
\begin{enumerate}
  \item 在针对移动相机的视频背景减除技术研究中,可以考虑利用视频图像帧之间的连续性减少冗余计算。例如本论文提出的算法中需要对每帧图像进行超像素分割。实际上相邻两帧图像中有很大部分区域是重合的,对这部分共同区域可以考虑利用前一帧超像素分割得到的结果,减少重复计算。还可以考虑引入时域连续的超像素分割算法~\cite{tsp},使得连续两帧图像的超像素分割结果在公共区域保持连续;
  \item 在视频背景减除算法中,考虑引入图像显著性检测算法对可能的前景进行预先判断,以提高前景检测精度并提高效率;
  \item 随着深度学习~\cite{DeepLearning}、神经网络~\cite{nerualNetworks}等人工智能方法的发展,基于这些方法的一些图像显著性检测,图像识别算法已经达到了业界领先的水平~\cite{NIPS2014_5547,DISC,ImageNet}。 本论文研究的视频和图像中的背景减除技术是人类视觉系统来与生俱来的功能。可以预见,在装备更强大的计算单元的基础上,辅以更加智能的人工智能算法,未来计算机快速识别视频图像中准确前景的能力或许将达到甚至超过人类的水平。
\end{enumerate}

